% LTeX: language=es
Este estudio ha demostrado la continua relevancia y eficacia de los filtros espaciales y morfológicos en diversas aplicaciones de procesamiento de imágenes. Los resultados obtenidos en la detección de tumores cerebrales, identificación de defectos en pantallas de celulares, detección de zonas urbanas en imágenes satelitales y segmentación de melanomas son particularmente prometedores. 

La capacidad de estas técnicas para delimitar con precisión áreas de interés, como la huella urbana en imágenes satelitales o la forma irregular de melanomas en imágenes dermatológicas, subraya su utilidad en escenarios complejos. Aunque se observaron algunas limitaciones, como la ocasional clasificación errónea en imágenes satelitales, el rendimiento general de estos métodos fue robusto y efectivo. 

Este trabajo resalta el valor persistente de los enfoques clásicos de procesamiento de imágenes, que pueden complementar eficazmente las técnicas más modernas como el aprendizaje profundo. Se recomienda continuar la investigación para refinar estas técnicas y explorar su integración con métodos más avanzados, lo que podría conducir a soluciones aún más potentes en el campo del análisis y procesamiento de imágenes.







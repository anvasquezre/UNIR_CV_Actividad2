%LTeX: language=es
Este trabajo explora la aplicación de filtros espaciales y morfológicos en diversos escenarios de procesamiento de imágenes, incluyendo imágenes satelitales, médicas e industriales. Se implementaron y evaluaron técnicas para la detección de 9 casos de estudio de interes comercial. Los resultados demuestran la eficacia de estos métodos en la delimitación precisa de áreas de interés y en la identificación de características específicas en diferentes tipos de imágenes. En particular, se logró una detección precisa de la huella urbana en imágenes satelitales y una segmentación efectiva de melanomas en imágenes dermatológicas. Aunque se observaron algunas limitaciones, como la ocasional clasificación errónea en ciertas condiciones, los métodos demostraron ser robustos y efectivos en general. Este estudio resalta la continua relevancia de los filtros espaciales y morfológicos en el procesamiento moderno de imágenes, sugiriendo su potencial como complemento valioso a técnicas más avanzadas como el aprendizaje profundo. Se concluye que estas técnicas clásicas siguen siendo herramientas poderosas en el análisis de imágenes y se recomienda su integración con métodos más recientes para desarrollar soluciones aún más efectivas en futuras aplicaciones.